\documentclass{apuntes}

\usepackage{tikztools}
\usepackage{tikz-3dplot}
\usepackage{textcomp}
\usepackage{tikz-qtree}
\usepackage{changepage}
\usepackage{listings}
% subfiguras
\usetikzlibrary{arrows}

\newcommand{\theauthor}{}
\newcommand{\thetitle}{Memoria P2\\Criptografía}
\newcommand{\rightheader}{Memoria P2}
\newcommand{\leftheader}{UAM - 2015/2016}

% Si no compila y el directorio tikzgen está creado, quitar estas dos sentencias.
%\precompileImages
%\precompileTikz

\title{Criptografía}
\author{Cristina Kasner Tourné\\Jose Antonio García del Saz}
\date{Curso 2015 - 2015 C1}

% Paquetes adicionales

% --------------------

\newcommand{\cte}{\text{Cte}}

\begin{document}
\pagestyle{plain}
\maketitle
%\abstract{Porque los apuntes de Chamizo son demasiado fáciles}

\tableofcontents
\newpage
\appendix
% Contenido.
\chapter{Ejercicios}
\section{Ejercicio 1-Seguridad perfecta}

En este ejercico nos piden implementar un programa que compruebe la seguridad perfecta.

Recordamos la definición de seguridad perfecta:

\begin{defn}[Seguridad Perfecta]
	Decimos que un criptosistema tiene seguridad perfecta si cumple:
	$$P_p(x|y) = P_p(x)$$
	Esto quiere decir que el conocimiento de texto cifrado no nos da ninguna información sobre el texto plano.
\end{defn}

Para esto hemos creado un fichero \textit{probabilidad.c} en el que implemetamos las funciones que calculan esas probabilidades.

La probabilidad $P_p(x)$ la calculamos recorriendo el fichero que tiene el texto plano y llevando la cuenta de las veces que aparece la letra i. 

\lstset{language=C, breaklines=true, basicstyle=\footnotesize}
\begin{lstlisting}
while((l=fgetc(f))!=EOF){
prob[l-65]++;
longText++;
}
\end{lstlisting}
Luego dividimos ese número entre la longitus del texto.

La probabilidad $P_p(x|y)$ la calculamos de la misma forma solo que contabilizando a la vez los caracteres del mensaje en plano y el mensaje cifrado.

\begin{lstlisting}
while((c=fgetc(cifrado))!=EOF){
p = fgetc(plano);
prob[p-65][c-65]++;
longText++;
}
\end{lstlisting}

Y finalmente volvemos a dividir entre la longitud del texto.

En la práctica podemos probar la seguridad perfecta con dos casos distintos.

\begin{itemize}
	\item claves equiprobables
	\item claves no equiprobables
\end{itemize}

Para generar las claves equiprobables utilizamos la función random de C.

Para las claves no equiprobables hemos escrito el siguiente código:


\begin{lstlisting}
for(i=0; i<2;i++){
if((clave>m/2)==0){
clave = rand() % m;
}else break;
}
\end{lstlisting}



De esta forma es mucho más probable que mi clave sea una clave menor que m/2 que mayor. 


Los resultados obtenidos tras probar el código son:
% % PANTALLAZO DE LOS RESULTADOS



\section{Ejercicio 2-Implementación del DES}

Para implementar el DES hemos creado un fichero que se llama \textit{funcionesDES.c} en el que están todas las funciones necesarias para el método.

La idea de las funciones de permutación es la siguiente:

Guardo el número que leo en la matriz de permutación, que es la posición del bit que va a ir en e 

Miro si ese bit es un 0 (positions[bit\%8] tiene un 1 en el bit que estoy mirando).

Si es un 0 no hago nada ya que he inicializado permutation a 0.

Si no es un cero meto en desired bit un 1 en la posicion apropiada y hago un 
XOR con el byte que ya hubiera en permutation, de forma que solo cambio el bit deseado


\section{Ejercicio X-Estudiar la linealidad de las cajas-S del AES}

Recordamos que una función lineal es aquella función $f$ que cumple que :
$$f(a + b) = f(a) + f(b)$$

Es sencillo comprobar que las cajas-S no cumplen esto. Hemos creado un sencillo programa al que le pasas dos bytes $a$ , $b$ y aplicamos la caja-S a cada uno de ellos y sumamos los resultados.

A su vez el programa suma $a + b$ y aplica la caja-S a dicha suma y vemos que los resultados son distintos.

La no linealidad de las cajas-S del AES es muy importante ya que las hacen resistentes al criptoanálisis lineal.

\printindex
\end{document}